\chapter{Resultados}

\section{Demandas computacionales de las simulaciones de lecturas largas}

Las simulaciones realizadas con VISOR generaron un total de 4.54 TB de datos 
sintéticos de secuenciación de lecturas largas mediante ONT. De este 
volumen total, aproximadamente 2.54 TB correspondieron a lecturas alineadas al 
genoma de referencia en formato BAM y sus respectivos ficheros índice, mientras 
que el volumen restante consistió en lecturas no alineadas en formato FASTQ 
(\textbf{Tabla~\ref{tab:file_sizes}}). Los datos de uso de CPUs, consumo de 
memoria RAM y los tiempos de ejecución se muestran en la 
\textbf{Figura~\ref{fig:hpc_visor_laser}}.

\begin{figure}[H]
    \centering
    \includegraphics[width=\textwidth]{data/cluster_bmk/hpc_visor_laser.pdf}
    \caption[Demanda de recursos computacionales del módulo VISOR LASeR]{Demanda 
    de recursos computacionales del módulo VISOR LASeR en función de la 
    longitud de lecturas. Los gráficos muestran el consumo de núcleos de CPU 
    (\textbf{a}), el uso de memoria RAM (\textbf{b}) y el tiempo de ejecución 
    (\textbf{c}) para longitudes de lectura de 15, 30, 50 y 100 Kb. Se 
    generaron tres réplicas técnicas WGS de tipo ``tumoral'' y una de tipo 
    ``normal'' para cada longitud, todas con una profundidad de secuenciación 
    de 40x. No se incluyen cálculos para el módulo VISOR HACk, ya que se 
    ejecuta una única vez durante pocos minutos con bajo consumo de recursos 
    para incorporar el conjunto de variantes estructurales en el genoma de 
    referencia que posteriormente alimentará a VISOR LASeR. Los datos en bruto 
    están disponibles en \url{https://github.com/villena-francis/tfe_unir/tree/main/data/cluster_bmk/hpc_data}.}
    \label{fig:hpc_visor_laser}
\end{figure}

\section{Detección de variantes estructurales}

\subsection{Exploración de los alineamientos de lecturas largas generados}

De las SVs incluidas en las instrucciones para la generación de los datos 
sintéticos (\textbf{Tabla~\ref{tab:stagesV1}}), se identificaron mediante 
inspección visual con GW los breakpoints asociados a la duplicación en tándem 
de 1q21 (\textbf{Figura~\ref{fig:chr1_del1q21}}), la inversión cromosómica de 6q25.1 
(\textbf{Figura~\ref{fig:chr6_inv6q25.1}}), la translocación de 
cortar-pegar de un fragmento de la región 9p21.3 (\textbf{Figura~\ref{fig:chr9_del9p21.3}}), 
la translocación recíproca entre 11q13 y 14q32 (\textbf{Figura~\ref{fig:chr11-14_translocation}}), 
y la deleción de 17p13 (\textbf{Figura~\ref{fig:chr17_del17p13}}). En relación a las 
SVs de mayor tamaño (la ganancia de 1q, la deleción de 17p y la pérdida de una 
copia del chr13), solo pudieron inferirse la pérdida del chr13 completo y la 
deleción del 17p mediante la representación gráfica de los valores 
de cobertura (\textbf{Figura~\ref{fig:chromosomes_comparison}}).

\begin{figure}[H]
    \centering
    \begin{subfigure}[b]{\textwidth}
        \centering
        \caption{chr1}
        \includegraphics[width=0.64\textwidth]{data/cluster_bmk/calls_data/chr1_40x_15000_A.png}
        \label{fig:chr1_subfig}
    \end{subfigure}
    
    \begin{subfigure}[b]{\textwidth}
        \centering
        \caption{chr13}
        \includegraphics[width=0.64\textwidth]{data/cluster_bmk/calls_data/chr13_40x_15000_A.png}
        \label{fig:chr13_subfig}
    \end{subfigure}
    
    \begin{subfigure}[b]{\textwidth}
        \centering
        \caption{chr17}
        \includegraphics[width=0.64\textwidth]{data/cluster_bmk/calls_data/chr17_40x_15000_A.png}
        \label{fig:chr17_subfig}
    \end{subfigure}
    
    \caption[Perfiles de profundidad de cobertura cromosómica]{Perfiles de 
    profundidad de cobertura cromosómica. (\textbf{a}) chr1: se observa la 
    amplificación de la región 1q21, mientras que el resto del cromosoma mantiene 
    cobertura normal excepto en la región centromérica; (\textbf{b}) chr13: la cobertura 
    reducida a aproximadamente la mitad (20x) refleja la pérdida de una copia 
    del cromosoma completo. La región 13p12 muestra cobertura nula debido a 
    duplicaciones segmentales (LCR13) que impiden el alineamiento único de las 
    lecturas; (\textbf{c}) chr17: la cobertura reducida a la mitad en 17p evidencia la 
    deleción de este brazo cromosómico. La región 17p12 muestra cobertura nula 
    debido a duplicaciones segmentales (CMT1A-REPs) que impiden el alineamiento 
    único de las lecturas. Los datos corresponden a la muestra tumoral A con 
    longitud media de lectura de 15 kb. Los gráficos fueron generados 
    utilizando la herramienta Wakhan.}
    \label{fig:chromosomes_comparison}
\end{figure}

\subsection{Formato de los arcivos de salida}

Los detectores de SVs generan archivos VCF similares a los producidos por los 
detectores de SNVs. Sin embargo, en estos casos cada fila representa uno de los 
puntos de ruptura (\textit{breakpoint}) que delimitan las SVs en cuestion. Las 
variantes estructurales simples, como inserciones o deleciones, requieren la 
identificación de dos puntos de ruptura, mientras que las variantes complejas, 
como las translocaciones recíprocas, necesitan la detección de cuatro puntos de 
ruptura. A modo ilustrativo, los archivos de salida finales generados por cada 
detector, correspondientes a una de las muestras simuladas están disponibles en 
\url{https://github.com/villena-francis/tfe_unir/tree/main/data/cluster_bmk/calls_data/raw}.

En el caso de \textbf{SAVANA}, se observa una limitación significativa en su 
archivo VCF de salida en comparación con otros detectores de variantes: su 
incapacidad para clasificar los tipos de variantes estructurales. Esta 
herramienta únicamente identifica y correlaciona puntos de ruptura sin 
proporcionar información sobre el tipo específico de variante estructural 
presente en cada caso.

Por su parte, \textbf{Severus} va más allá de la detección convencional de 
variantes estructurales al incorporar un procesamiento especializado para 
repeticiones en tándem de número variable (VNTRs, del inglés \textit{Variable 
Number Tandem Repeats}), lo que permite una anotación precisa de las variantes 
estructurales dentro de estas regiones. La herramienta proporciona una salida 
integral que incluye archivos VCF estándar, métricas detalladas de calidad en 
archivos de registro (\textit{log}), y representaciones gráficas de los 
reordenamientos cromosómicos mediante gráficos de visualización.

En cuanto a \textbf{Sniffles2}, este detector genera exclusivamente archivos 
VCF estándar, sin archivos de soporte adicionales ni visualizaciones 
complementarias. No obstante, al igual que Severus, también incorpora 
funcionalidades para mejorar la detección en regiones con VNTRs para optimizar 
la precisión en la detección de variantes asociadas a estas regiones del genoma.

Finalmente, \textbf{SVision-pro}, a pesar de utilizar un enfoque innovador 
basado en codificación de imágenes para analizar características genómicas y 
detectar variaciones inter-genómicas entre muestras pareadas tumor-normal, 
limita su salida a archivos VCF estándar en los que las variantes no aparecen 
ordenadas secuencialmente por cromosoma ni por sus coordenadas genómicas. Las 
representaciones visuales empleadas en su pipeline de procesamiento interno no 
son accesibles al usuario como ficheros de salida.  

\subsection{Rendimiento}

El análisis comparativo del rendimiento de los cuatro métodos de detección de 
variantes estructurales reveló diferencias significativas en términos de precisión, 
sensibilidad y tasa de falsos positivos. 

\begin{figure}[H]
    \centering
    \includegraphics[width=\textwidth]{data/cluster_bmk/perform_calls.pdf}
    \caption[Rendimiento de los métodos de detección de variantes estructurales 
    evaluados]{Rendimiento de los métodos de detección de variantes estructurales 
    SAVANA, Severus, Sniffles2 y SVision-pro, basado en las métricas de precisión, 
    exhaustividad (\textit{recall}) y puntuación F1. Para cada profundidad de 
    secuenciación, se obtuvieron tres resultados por métrica mediante la comparación 
    de una muestra normal contra tres réplicas técnicas de la muestra tumoral. Los 
    datos en bruto utilizados para todos los cálculos están disponibles en 
    \url{https://github.com/villena-francis/master_thesis/tree/main/data/cluster_bmk/calls_data/callings.csv}.}
    \label{fig:perform_calls}
\end{figure}

\textbf{SAVANA} demostró el mejor desempeño global con un F1-score promedio de 0.977 ± 0.041, 
manteniendo una precisión del 95.8\% y una sensibilidad del 100\%, detectando 
consistentemente las cinco SVs esperadas (duplicación de 1q21, inversión de 6q25.1, 
translocación de 9p21.3, translocación recíproca 11q13-14q32 y deleción de 17p13) 
en todas las réplicas y longitudes de lectura evaluadas, con un promedio de solo 
0.25 falsos positivos por análisis. 

\textbf{Severus}, aunque alcanzó una sensibilidad perfecta (100\%) detectando todas las 
SVs esperadas, presentó una precisión notablemente inferior (21.8\% ± 5.8\%) 
debido a una alta tasa de falsos positivos (promedio de 19.25 por análisis), 
resultando en un F1-score de 0.355 ± 0.076. 

\textbf{SVision-pro} detectó correctamente cuatro de las cinco SVs (inversión de 6q25.1, 
translocaciones de 9p21.3 y 11q13-14q32, y deleción de 17p13), alcanzando una 
sensibilidad del 80\%, pero su rendimiento estuvo severamente comprometido por 
un número excepcionalmente elevado de falsos positivos (promedio de 546.2), lo 
que redujo drásticamente su precisión a 0.7\% ± 0.1\% y su F1-score a 0.014 ± 0.001. 

\textbf{Sniffles2} mostró el peor desempeño general, detectando únicamente la translocación 
recíproca 11q13-14q32 en el 50\% de las réplicas, con una sensibilidad promedio 
del 10\% ± 10.4\%, precisión del 2.1\% ± 2.4\%, y F1-score de 0.035 ± 0.038. 

Estos resultados indican que SAVANA ofrece el mejor balance entre sensibilidad 
y precisión para la detección de SVs en datos de secuenciación de lectura larga, 
mientras que los otros métodos requieren optimización adicional de parámetros o 
filtrado post-procesamiento para reducir los falsos positivos y mejorar su 
aplicabilidad clínica.

\newpage

\subsection{Demandas computacionales}

El análisis de demanda computacional reveló diferencias sustanciales entre los 
cuatro métodos de detección de SVs en términos de tiempo de ejecución, uso de 
memoria RAM y eficiencia del uso de CPU \textbf{Figura~\ref{fig:hpc_calls}}. 

\begin{figure}[H]
    \centering
    \includegraphics[width=\textwidth]{data/cluster_bmk/hpc_callers.pdf}
    \caption[Demanda de recursos computacionales para métodos de detección de 
    variantes estructurales]{Demanda de recursos computacionales para los métodos 
    de detección de variantes estructurales SAVANA, Severus, Sniffles2 y 
    SVision-pro, medida en núcleos de CPU, memoria RAM (GB) y tiempo de ejecución 
    (minutos). Para cada longitud media de lectura, se obtuvieron tres 
    mediciones por métrica mediante la comparación de una única muestra normal 
    contra tres réplicas técnicas de la muestra tumoral. Los datos en bruto 
    completos utilizados para todos los cálculos están disponibles en 
    \url{https://github.com/villena-francis/master_thesis/tree/main/data/cluster_bmk/hpc_data}.}
    \label{fig:hpc_calls}
\end{figure}



Una distinción clave en la asignación de unidades de procesamiento radica en que 
el modelo de detección de SVision-pro, basado en redes neuronales, está 
optimizado para GPU, mientras que SAVANA, Severus y Sniffles2 dependen de 
computación basada en CPU. Para las ejecuciones de SVision-pro se asignaron 24 
núcleos de CPU junto con recursos de GPU, dado que pruebas preliminares 
mostraron que la utilización máxima se mantenía por debajo de 20 núcleos. 

\textbf{SVision-pro} presentó el menor tiempo de ejecución promedio (11.0 ± 3.7 
minutos) con un consumo de memoria moderado (12.14 ± 0.94 GB), lo que refleja 
la ventaja computacional de la aceleración por GPU, aunque mostró una utilización 
de CPU del 66.1\% ± 22.5\% de los recursos asignados. 

\textbf{Sniffles2} demostró ser el método más eficiente en términos de memoria 
RAM entre los métodos basados en CPU, utilizando únicamente 5.57 ± 0.78 GB con 
24 cores, alcanzó el segundo menor tiempo de ejecución (18.5 ± 3.8 minutos) y 
presentó la menor utilización de CPU (38.1\% ± 9.4\%), lo que indica una mejor 
optimización del uso de recursos. 

\textbf{Severus} exhibió un consumo de memoria bajo (8.20 ± 1.56 GB) con 24 cores, 
un tiempo de ejecución de 28.1 ± 5.1 minutos y una utilización de CPU moderada 
(53.5\% ± 13.4\%). 

\textbf{SAVANA}, a pesar de presentar el mayor tiempo de ejecución (29.7 ± 4.6 
minutos) y la demanda de memoria más elevada (40.51 ± 10.52 GB, con picos de 
hasta 61.86 GB) utilizando 24 cores, mantuvo una utilización de CPU del 63.2\% 
± 9.8\%. 

El análisis por longitud de lectura indicó que el tiempo de ejecución de SAVANA 
aumentó progresivamente desde 24.8 minutos (15 kb) hasta 33.1 minutos (100 kb), 
mientras que SVision-pro mostró el patrón inverso, con tiempos menores para 
lecturas más largas (8.7 minutos para 50 kb).
