\chapter{Discusión}

\section{Parámetros de las simulaciones}

La simulación de variantes estructurales se configuró para representar 
secuenciación masiva (\textit{bulk sequencing}) de una población celular 
homogénea que contiene un único clon tumoral, estableciendo todas las frecuencias 
alélicas de variante (VAFs, del inglés \textit{Variant Allele Frequencies}) en 
0,5. Esta configuración garantiza que cada variante esté presente en un alelo 
y representada en aproximadamente la mitad de las lecturas de secuenciación 
generadas. Este diseño buscaba proporcionar una representación clara de las 
variantes en las lecturas, asegurando una detección fiable por parte de los 
métodos evaluados y facilitando la validación visual de los breakpoints mediante 
inspección con GW. Sin embargo, este escenario idealizado difiere sustancialmente 
de las muestras tumorales reales, donde las frecuencias alélicas varían 
considerablemente debido a la heterogeneidad intratumoral, la evolución clonal 
y la contaminación con células normales \cite{dagogo-jack_tumour_2018}. Esta 
simplificación, aunque permite una evaluación comparativa rigurosa en 
condiciones controladas, limita la extrapolación directa de los resultados 
a muestras clínicas reales.

Cabe destacar que el protocolo de la plataforma PromethION para ADN humano de 
10 kb con el kit de secuenciación por ligación V14 tiene como objetivo generar 
una cobertura genómica de aproximadamente 30-40x con una única celda de flujo 
\cite{noauthor_ligation_2022}. Por esta razón, se concluyó que, al menos para 
alteraciones \textit{driver}, la profundidad de secuenciación de 40x 
resultaba coste-eficiente, estableciéndose como valor 
estándar para el presente trabajo.

La selección de SVs para la simulación estuvo influenciada por la investigación 
en curso del grupo de lecturas largas de la Unidad de Bioinformática del CNIO, 
específicamente por su colaboración con el grupo de Neoplasias Hematológicas 
del Hospital 12 de Octubre en la secuenciación WGS de muestras de pacientes 
con MM, comparando muestras pretratamiento y recaídas. En consecuencia, el 
conjunto de SVs incluye tanto variantes características de MM como SVs más 
especulativas relacionadas con cáncer, con el objetivo de cubrir un espectro 
más amplio de variantes estructurales. A pesar del uso generalizado de GRCh38 
en la investigación, principalmente motivado y mantenido por el uso de 
plataformas de secuenciación de lecturas cortas, se eligió el genoma de 
referencia T2T debido a su mayor completitud y continuidad en regiones 
previamente difíciles de ensamblar.

Cabe destacar que, para el caso de la amplificación del brazo cromosómico 1p, 
lo más probable es que haya sido un error derivado de la documentación de VISOR-LASeR
[CITA], la cual indica que \textit{``[...] for tandem duplication and inverted tandem 
duplication, must be an integer, specifying the number of times the region have 
to be duplicated''}. Esto sugiere que el valor 1 debería generar una
duplicación (dos copias); sin embargo, en su ejecución simplemente mantiene una copia
del par cromosómico seleccionado. Por esta razón se observa en el perfil de
cobertura del chr1 una ganancia de 3 copias de la región 1q21 en lugar de las
4 especificadas en las instrucciones para el simulador (\textbf{Figura~\ref{fig:chr1_subfig}}).

\section{Rendimiento y limitaciones de los detectores de SVs}

Las métricas de rendimiento posicionan a SAVANA como la herramienta con mejor 
desempeño. Resulta notable que SAVANA alcance este rendimiento sin requerir un 
archivo de VNTRs, a diferencia de otras herramientas evaluadas que dependen de 
esta información adicional. Sin embargo, su limitación a la correlación de puntos 
de ruptura sin clasificación de SV representa un inconveniente significativo. Esta 
limitación, combinada con un consumo de RAM bastante superior, impacta negativamente
su utilidad práctica. 

Severus iguala la sensibilidad de SAVANA mientras mantiene un menor uso 
de RAM y proporciona clasificación exhaustiva de SV junto con representaciones 
visuales de los reordenamientos cromosómicos. Estas características probablemente 
influyeron en la integración de Severus para la detección de variantes somáticas
a través de EPI2ME, la plataforma de código abierto de Oxford Nanopore 
diseñada para proporcionar a los científicos de laboratorio húmedo una interfaz 
amigable para el análisis de datos sin requerir habilidades avanzadas de 
bioinformática. Sin embargo, la documentación de EPI2ME carece de análisis 
comparativos que justifiquen la selección de Severus sobre otros detectores de SV, 
posiblemente debido a su enfoque en la accesibilidad práctica para su audiencia 
objetivo en lugar de la evaluación técnica comparativa
\cite{oxford_nanopore_technologies_epi2me_nodate}. Los falsos positivos que 
redujeron la precisión de Severus tienen un perfil bastante concreto: inserciones 
de entre 50-60 pb y deleciones de 2-0,5 kb. Una optimización adicional de 
parámetros o un filtrado post-procesamiento podría reducir estos falsos positivos 
y mejorar su aplicabilidad.

La sensibilidad de Svision-pro fue ligeramente inferior respecto a SAVANA y 
Severus, aunque su tiempo de ejecución fue notablemente menor gracias a la 
aceleración por GPU. No obstante, la gran cantidad de falsos positivos supone un 
problema; al igual que SAVANA, este método carece del uso de coordenadas de VNTRs 
para mejorar su desempeño y muy posiblemente se vea beneficiado al implementarlo, 
aunque como alternativa un filtrado post-procesamiento también podría mitigar el 
problema.

Sniffles2 mostró muy bajo rendimiento, siendo únicamente capaz de detectar 
la translocación recíproca 11q13-14q32 en la mitad de las muestras tumorales 
generadas. No obstante, también detectó como falsos positivos translocaciones entre
otros cromosomas. Cabe destacar que este método hace una inferencia de variantes 
somáticas, al que denomina modo mosaico, haciendo un análisis exclusivo de la muestra
tumoral, a diferencia de los tres métodos anteriores que usan pares tumor-normal.
De haber tenido buen desempeño, habría sido una opción coste-eficiente al no 
necesitar secuenciación de muestra normal. También es necesario destacar que
los análisis de este trabajo emplearon conjuntos mínimos de argumentos para todos 
los detectores, confiando en los parámetros por defecto configurados por los 
desarrolladores para ajustes complejos. Esta aproximación estandarizada podría 
explicar el bajo rendimiento de Sniffles2, ya que la configuración por defecto 
podría no estar optimizada para variantes tan extensas como las evaluadas en este
trabajo.

\section{Relevancia clínica y desafíos técnicos}

El conjunto de herramientas VISOR demostró ser valioso para evaluar la capacidad 
de los detectores de SV para identificar eventos estructurales de gran tamaño 
característicos del mieloma múltiple utilizando lecturas largas de ONT, 
particularmente aquellos con significancia diagnóstica. A través de la generación 
de datos sintéticos, se validó exitosamente la detección de algunas aberraciones 
cromosómicas frecuentes en mieloma múltiple. Estas variantes estructurales, 
actualmente verificadas en entornos clínicos mediante FISH debido a las limitaciones 
del ensamblaje de secuenciación de lecturas cortas, representan marcadores 
diagnósticos críticos que potencialmente podrían identificarse mediante enfoques 
de secuenciación de lecturas largas \cite{garces_time_2026}.

Los datos simulados incluyeron SVs distribuidas como haplotipos (h1 y h2); sin
embargo, el genoma de referencia con el que se generaron las lecturas presenta
una única copia de cada cromosoma, lo cual deja al resto de regiones sin SVs sin
variabilidad suficiente para hacer \textit{haplotype phasing}, es decir, 
la asignación de variantes al mismo cromosoma parental para diferenciarlos. En
muestras reales de pacientes con mieloma múltiple, esta información permite 
reconstruir y analizar aberraciones cromosómicas (CNAs), mitigando las carencias de
los métodos de detección de SVs respecto a la ganancia o pérdida de segmentos o 
cromosomas completos. Otra característica genómica importante es la pérdida de 
heterocigosidad con número de copias neutral (CN-LOH, del inglés \textit{copy-neutral 
loss of heterozygosity}). En MM, por ejemplo, las regiones pueden parecer diploides 
pero ser homocigotas debido a la deleción de un alelo y la duplicación del otro 
\cite{garces_time_2026} (\textbf{Figura~\ref{fig:FISH_and_CNAs}}). 

Los intentos iniciales de visualización de los datos sintéticos se realizaron 
utilizando el ampliamente adoptado \textit{Integrative Genomics Viewer} (IGV); 
sin embargo, esta herramienta no permitió la exploración de los archivos completos 
de lecturas alineadas (BAM), mostrando tiempos de carga lentos y numerosas caídas 
del programa. Una alternativa podría haber sido leer los VCFs y, en base a los 
hallazgos, extraer pequeñas regiones que contengan sus respectivos puntos de 
ruptura; no obstante, esto habría eternizado los flujos de trabajo para validar 
visualmente las variantes. GW surgió como una alternativa capaz, ya que permite 
procesar y navegar por todo el genoma en cada fichero sin tiempos de carga al 
desplazarse entre las diferentes SVs anotadas en los VCFs (\textbf{Figura~\ref{fig:GW_UI}}). 
Cabe resaltar que esto fue posible haciendo \textit{streaming} de los archivos 
desde el clúster del CNIO sin necesidad de descargar ningún archivo directamente 
a la computadora local desde la que se ejecutó GW; esto supone una enorme 
agilización en el proceso de inspección y validación de lecturas genómicas en el 
contexto de un gran número de muestras.

\section{Líneas futuras}

El desarrollo de herramientas robustas para el análisis de SVs en genomas 
cancerosos requiere pruebas y validación extensivas. Si bien este estudio evaluó 
las capacidades de detección de SVs utilizando datos sintéticos con diferentes 
longitudes medias de lecturas, principalmente demuestra el valor de generar 
conjuntos de datos simulados para la evaluación comparativa de herramientas 
existentes. La implementación de un genoma de referencia diploide como el 
T2T-YAO \cite{he_t2t-yao_2023}, que incluye ensamblajes T2T completos de ambos 
haplotipos para los 22 autosomas y los cromosomas sexuales (X e Y), además 
del genoma mitocondrial, permitiría realizar fasado haplotípico (\textit{haplotype 
phasing}) de las lecturas sintéticas generadas, facilitando la distinción entre 
variantes somáticas y germinales. Estos avances computacionales contribuyen en 
última instancia a mejorar el análisis genómico del cáncer en datos humanos reales.

Es necesario trasladar la experiencia adquirida con los datos sintéticos al 
análisis de datos reales. Los métodos SAVANA y Severus mostraron los mejores
resultados y, con ligeros ajustes para este último, podrían usarse para la 
detección confiable de SVs. No obstante, la experiencia adquirida tras haber 
analizado ya algunas muestras de mieloma múltiple indica que el ruido biológico 
y el elevado incremento en el número de hallazgos aumentan notablemente las 
demandas computacionales, volviendo a ser el punto más crítico las elevadas 
demandas de memoria RAM, llegando a superar los 300 GB.

Existen eventos de variación estructural relevantes que no son fácilmente 
identificables mediante la inspección visual de alineamientos ni detectables 
por los métodos convencionales de detección de SVs. Para estos casos, resulta 
necesario complementar el análisis con herramientas especializadas en la 
inferencia de aberraciones en el número de copias (CNAs) para datos de secuenciación
de lecturas largas, como Wakhan, cuyo desarrollo y optimización continúa en 
proceso de desarrollo y maduración.