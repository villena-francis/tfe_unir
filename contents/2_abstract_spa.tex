\textcolor{UNIR}{\fontspec{Calibri Light}\fontsize{18}{28}\selectfont
Resumen}

Las variantes estructurales (SV) constituyen modificaciones genómicas que incluyen 
deleciones, inserciones y reordenamientos de fragmentos, cuyo tamaño varía desde 
kilobases hasta cromosomas enteros. A pesar de su relevancia como biomarcadores en 
patologías oncológicas, estas variantes han sido poco estudiadas en comparación con 
las variantes de nucleótido único, principalmente debido a las restricciones propias 
de las tecnologías de secuenciación de lectura corta que han predominado en los 
proyectos de secuenciación genómica a gran escala. Esta situación ha experimentado 
una transformación significativa con el surgimiento de las tecnologías de 
secuenciación de lectura larga, que han hecho posible obtener el primer genoma de 
referencia humano completamente ensamblado, de telómero a telómero, resolviendo 
exitosamente las regiones que las lecturas cortas no lograban cubrir. Este proyecto 
tiene como objetivo realizar una evaluación integral del desempeño de los 
identificadores de variantes estructurales basados en lecturas largas, 
particularmente en el contexto del estudio de la evolución tumoral. Para abordar 
la escasa disponibilidad de conjuntos de datos apropiados, hemos diseñado flujos 
de trabajo especializados que utilizan recursos computacionales de alto rendimiento 
para producir datos sintéticos con SV personalizadas, lo cual facilita una 
comparación robusta de diferentes métodos de detección de variantes estructurales. 
Esta estrategia computacional posibilita la evaluación sistemática de algoritmos 
de detección de SV bajo condiciones controladas, ofreciendo información relevante 
sobre su desempeño y confiabilidad.

\vspace{0.5cm}

\noindent\textbf{Palabras clave:} 
    
Variantes Estructurales, Genómica del Cáncer, Llamada de Variantes, 
Secuenciación por Nanoporos, Lecturas Largas.

