\chapter{Objetivos}

Este proyecto se centró en la evaluación de enfoques mediante secuenciación 
de lectura larga por nanoporos para rastrear y analizar variaciones 
estructurales somáticas a lo largo de la evolución de genomas oncológicos. Con 
este fin, se establecieron los siguientes objetivos específicos:

\begin{enumerate}
    \item Evaluar la capacidad de los datos de secuenciación WGS de lecturas 
    largas obtenibles con una única celda de flujo de ONT PromethION, en 
    combinación con diferentes longitudes medias de lecturas, para la 
    reconstrucción de variantes estructurales de gran tamaño que abarcan 
    bandas cromosómicas completas mediante alineamiento a un genoma T2T.

    \item Comparar detectores de variantes estructurales actuales considerando 
    precisión de detección y eficiencia computacional, con especial atención 
    a capacidades de clasificación y demandas de recursos, para identificar 
    soluciones óptimas en flujos de trabajo de análisis rutinarios.

    \item Explorar el potencial de la secuenciación de lectura larga por 
    nanoporos para detectar marcadores genómicos relevantes en Mieloma 
    Múltiple más allá de los tradicionalmente identificados por FISH, y 
    evaluar estrategias complementarias para mejorar la caracterización de 
    alteraciones cromosómicas.
\end{enumerate}