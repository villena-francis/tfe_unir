\chapter{Material y métodos}

\section{Recursos computacionales}

El desarrollo del código para este proyecto se llevó a cabo en un MacBook Pro 
con un procesador M4 Pro, que incluye una CPU de 14 núcleos, una GPU de 20 
núcleos, 24 GB de memoria unificada y 1 TB de almacenamiento SSD.

Debido a las demandas computacionales de las tareas requeridas para lograr los 
objetivos propuestos, se utilizó el clúster de computación de alto rendimiento 
(HPC) del CNIO. Esta infraestructura cuenta actualmente con 12 nodos de cómputo 
cuyas configuraciones se detallan en la \textbf{Tabla~\ref{tab:nodes}}.

\vspace{0.50cm}

\begingroup
\footnotesize
\setlength{\LTpre}{0pt}
\setlength{\LTpost}{0pt}
\begin{longtable}{>{\RaggedRight\arraybackslash}p{1.75cm} 
                  >{\RaggedRight\arraybackslash}p{2.75cm} 
                  >{\RaggedRight\arraybackslash}p{2.25cm}
                  >{\RaggedRight\arraybackslash}p{2cm}
                  >{\RaggedRight\arraybackslash}p{4cm}}
    \caption[Especificaciones técnicas de los nodos del clúster HPC del CNIO]{Especificaciones técnicas 
    del los nodos del clúster HPC del CNIO \cite{tomas_di_domenico_cluster_nodate}.}\vspace{0.30cm}\label{tab:nodes}\\
    \toprule
    \textbf{Cantidad} & \textbf{Identificadores} & \textbf{núcleos CPU} & \textbf{RAM} & \textbf{GPUs} \\ 
    \midrule
    \endfirsthead
    
    \multicolumn{5}{@{}l}{\RaggedRight \textbf{\tablename\ \thetable{}} -- Continued} \\
    \\
    \toprule
    \textbf{Cantidad} & \textbf{Identificadores} & \textbf{núcleos CPU} & \textbf{RAM} & \textbf{GPUs} \\ 
    \midrule
    \endhead
    \\
    \midrule 
    \multicolumn{5}{r}{Continued on next page} \\
    \endfoot
    
    \bottomrule
    \endlastfoot
    
    \\
    1     & bc001      & 24  & 32 GB  & -- \\
    \\
    6     & bc00[2-7]  & 52  & 512 GB & -- \\
    \\
    3     & bc00[8-10] & 128 & 1 TB   & -- \\
    \\
    1     & hm001      & 224 & 2 TB   & -- \\
    \\
    1     & gp001      & 112 & 768 GB & 3 x Nvidia A100 80 GB \\
    \\
    
\end{longtable}
\endgroup

Los recursos de almacenamiento del clúster incluyen 52 TB de espacio estándar 
para los directorios de inicio de los usuarios, además de 512 TB de 
almacenamiento de alto rendimiento optimizado para las operaciones de entrada y 
salida de los trabajos de cálculo. De este almacenamiento de alto rendimiento, 
se asignaron específicamente 15 TB para la ejecución de código y la generación 
de datos asociados a este proyecto.

\section{Herramientas de software}

Positron fue utilizado como interfaz principal para el acceso al clúster a través 
de protocolo SSH (\textit{Secure Shell}) y el desarrollo de código. Se utilizaron 
principalmente los lenguajes de programación Bash, Python y R.

El clúster opera bajo el gestor de tareas Slurm, un sistema basado en Linux/Unix 
para la gestión de recursos de HPC. La integración de los flujos de trabajo se 
logró mediante Snakemake, un administrador que permite ejecutar análisis 
reproducibles y escalables a través de la definición de reglas basadas en Python. 
La instalación del software para cada regla se realizó mediante ambientes Conda, 
lo que evita instalar software en cada nodo del clúster donde se ejecuten los 
trabajos y posibles problemas de compatibilidad entre los diferentes programas. 
Para gestionar los ambientes Conda, se utilizó Miniforge, una distribución 
ligera a la que se incluyó el repositorio Bioconda para descargar herramientas 
especializadas en bioinformática.

En la \textbf{Tabla~\ref{tab:software}} se recogen todas las herramientas de 
software utilizadas en este trabajo. En las siguientes secciones de este apartado 
se describe cómo se han integrado las distintas herramientas para crear los 
flujos de trabajo necesarios para alcanzar los objetivos específicos de este 
proyecto. 

\subsection{Simulación de datos de secuenciación de lecturas largas con eventos
estructurales y detección de variantes}

\subsubsection{Simulador de datos}

Para la simulación específica de haplotipos con variantes estructurales simples 
y complejas, se seleccionó el set de herramientas VISOR. Su módulo VISOR 
LASeR incluye un modelo de incorporación de errores entrenado con lecturas de ONT 
R10.4.1, proporcionado por Badread \cite{wick_badread_2019}.

\subsubsection{Detectores de variantes estructurales}

Se seleccionó un conjunto de herramientas de detección de SV en base a dos 
criterios clave: compatibilidad con lecturas largas de ONT y capacidad de 
detección de SV somáticas. Cada herramienta seleccionada presenta 
características distintivas e interesantes para los análisis:

\begin{itemize}

    \item \textbf{SAVANA}: implementa un modelo de aprendizaje automático 
    entrenado con muestras pareadas tumor-normal para detectar SV somáticas y 
    aberraciones en el número de copias (CNAs) en casos clínicos.

    \item \textbf{Severus}: especializado en análisis comparativos tumor/normal, 
    admite múltiples muestras tumorales y emplea marcos de grafos de puntos de 
    ruptura para la detección de reordenamientos cromosómicos complejos.

    \item \textbf{Sniffles2}: una herramienta pionera de detección de SV en 
    lecturas largas de ONT desde 2018, manteniéndose en desarrollo continuo y con 
    actualizaciones regulares.

    \item \textbf{SVision-pro}: emplea un enfoque basado en redes neuronales que 
    convierte características genómicas de muestras pareadas en representaciones 
    de imágenes para la detección comparativa de SV.

\end{itemize}

\subsubsection{Flujo de trabajo para la generación y detección de variantes 
estructurales}

Se desarrolló un flujo de trabajo integral para simulaciones basadas en VISOR y 
el posterior análisis de detección de variantes estructurales. Las etapas principales
del proceso se muestran en la \textbf{Figura \ref{fig:visor-sim}}. El código fuente 
y la documentación asociada están disponibles en el siguiente repositorio: 
\url{https://github.com/villena-francis/visor-simulations}.

\begin{figure}[!htbp]
    \centering
    \includegraphics[scale=1.25]{img/visor-simulations.pdf}
    \captionsetup{labelfont=bf, textfont={normalsize,it}}
    \caption[Versión simplificada del flujo de trabajo ``visor-simulations'' para 
    la simulación de lecturas largas y la detección de SVs]{Versión simplificada 
    del flujo de trabajo ``visor-simulations'' para la simulación de lecturas 
    largas y la detección de SVs. VISOR-HACk genera archivos FASTA (secuencias
    molde) con SVs incorporadas utilizando un genoma de referencia e
    instrucciones de haplotipos en formato BED (coordenadas genómicas delimitadas 
    por tabulaciones). El script makeBED crea un archivo BED (intervalos genómicos) 
    a partir de los tamaños máximos de cromosomas extraídos de los FASTA de haplotipos. 
    Posteriormente, VISOR-LASeR genera archivos BAM (lecturas de secuenciación 
    alineadas) junto con sus índices (.bai), empleando como entrada los ficheros
    obtenidos en los pasos anteriores. Finalmente, las lecturas alineadas son 
    procesadas por los detectores de SVs, generando cada uno su correspondiente 
    archivo VCF (Variant Call Format) con las SVs identificadas.}
    \label{fig:visor-sim}
\end{figure}

\subsubsection{Configuración de las simulaciones}

En cada simulación de secuenciación de genoma completo mediante ONT, se 
introdujeron SVs en el genoma de referencia T2T-CHM13 (\textit{chm13v2.0\_maskedY\_rCRS.fa}), 
para generar lecturas sintéticas con una cobertura media de 40x. Con el objetivo de 
evaluar el efecto de la longitud de las lecturas sobre la eficiencia del mapeo al genoma de 
referencia y la capacidad de detección de las SVs introducidas, se realizaron 
simulaciones con cuatro longitudes medias de lecturas: 15, 30, 50 y 100 kb. Para 
cada longitud, se generó una muestra normal (sin SVs) y tres réplicas tumorales, 
generándose así 12 muestras en total.

El conjunto de SVs utilizadas se basó en aberraciones cromosómicas características 
de mieloma múltiple, algunas descritas en la literatura científica y otras 
observadas en pacientes reales (\textbf{Tabla~\ref{tab:stagesV1}}). Dado que los 
puntos de rotura específicos no se encuentran detallados en la literatura 
clínica para el genoma de referencia T2T-CHM13, sus coordenadas se determinaron 
de forma aproximada utilizando el navegador genómico UCSC Genome Browser.

\begingroup
\vspace{0.50cm}

\footnotesize

\begin{longtable}{>{\RaggedRight\arraybackslash}p{2.25cm} 
                  >{\RaggedRight\arraybackslash}p{10cm} 
                  >{\RaggedLeft\arraybackslash}p{2cm}}
    \captionsetup{labelfont=bf, textfont={normalsize,it}}
    \caption[Características de las SVs simuladas]{Características de las SVs 
    simuladas, inspiradas en datos de MM. \cite{aksenova_genome_2021}. Los valores 
    de tamaño representan la longitud final en pares de bases (pb) tras la inserción en 
    el genoma, aquellos marcacados con * implican puntos de rotura en regiones centroméricas. La 
    duplicación en tándem consiste en un fragmento de 2 030 586 pb 
    repetido cuatro veces. Los archivos de entrada para la simulación de estas
    lecturas con \textit{visor-simularions} están disponibles en 
    \url{https://github.com/villena-francis/tfe_unir/tree/main/data/cluster_bmk/visor_v1}.}
    \vspace{0.3cm}\label{tab:stagesV1}\\

    \toprule
    \textbf{Tipo de SV} & \textbf{Descripción} & \textbf{Tamaño (pb)} \\ 
    \midrule
    \endfirsthead
    
    \multicolumn{3}{@{}l}{\RaggedRight \textbf{\tablename\ \thetable{}} -- Continued} \\
    \\
    \toprule
    \textbf{SV type} & \textbf{Description} & \textbf{Size (pb)} \\ 
    \midrule
    \\
    \endhead
    \\
    \midrule 
    \multicolumn{3}{r}{\footnotesize Continued on next page} \\
    \endfoot
    
    \bottomrule
    \endlastfoot

    \\
    Duplicación en tandem             & Ganancia de un brazo 1q extra, una de las anomalías citogenéticas estructurales más frecuentes en el MM. & 123600854* \\
    \\
    Duplicación en tandem             & Amplificación de 1q21, otra de las anomalías estructurales asociadas a progresión de MM.                 & 11997705 \\
    \\
    Deleción                          & Deleción del 17p (~10\% de casos), indicador de mal pronóstico. La región mínima delecionada en 17p13 incluye el gen supresor tumoral \textit{TP53}. & 4307895 \\
    \\
    Deleción                          & Deleción total del 17p, también observado en células con MM.   & 23892418* \\
    \\
    Deleción                          & Perdida de una copia completa del cromosoma 13, también frecuente en MM  &  113566675 \\
    \\
    Translocación recíproca           & Reordenamiento cromosómico que afecta al gen de la cadena pesada de inmunoglobulina (\textit{IGH}) en 14q32, la más frecuente es con 11q13.                                                 & 500000  \\
    \\
    Translocación (cortar-pegar)      & Reordenamiento que afecta a \textit{CDKN2A}, gen supresor tumoral crucial cuya inactivación mediante mutaciones o deleciones es una de las alteraciones más frecuentes en cánceres humanos, solo superada por alteraciones en \textit{TP53}. & 50000 \\
    \\
    Inversión                         & Inversión cromosómica en 6q25.1, incluida como variante control para validar las capacidades de los detectores de SVs, aunque no es típicamente característica en mieloma múltiple.          & 3600000  \\
    \\

\end{longtable}
\endgroup

\subsection{Análisis comparativo de detectores de variantes estructurales}

\subsubsection{Marco de evaluación}

La evaluación del rendimiento se basó en un marco de clasificación binaria 
que categorizó los resultados de detección de SVs como:

\begin{itemize}
    \item \textbf{Verdaderos Positivos (VP)}: SVs simuladas que fueron detectadas 
    correctamente.

    \item \textbf{Falsos Positivos (FP)}: SVs identificadas por el detector pero 
    ausentes en la simulación, verificadas mediante inspección de lecturas.

    \item \textbf{Falsos Negativos (FN)}: SVs simuladas presentes en las lecturas 
    pero no detectadas por el detector.
\end{itemize}

Los Verdaderos Negativos (VN) fueron excluidos de este análisis debido a su falta 
de aplicabilidad en la evaluación de la detección de SVs, dado el extenso 
espacio genómico y la naturaleza de los métodos de detección de variantes 
estructurales.

\subsubsection{Métricas de rendimiento}

Dada la imposibilidad de cuantificar los VN, se seleccionaron métricas 
independientes de esta categoría para evaluar el rendimiento de los detectores
de SVs:

\begin{itemize}
    \item \textbf{Sensibilidad} (\textit{Recall}): Proporción de casos positivos correctamente 
    identificados
    \begin{equation}
        \label{eq:recall}
        Recall = \frac{VP}{VP + FN}
    \end{equation}

    \item \textbf{Precisión} (\textit{Precision}): Exactitud de las predicciones positivas
    \begin{equation}
        \label{eq:precision}
        Precision = \frac{VP}{VP + FP}
    \end{equation}

    \item \textbf{Puntuación F1} (\textit{F1 Score}): Media armónica de la precisión (P) y la sensibilidad (S)
    \begin{equation}
        \label{eq:f1score}
        F1 = 2 \cdot \frac{P \cdot S}{P + S}
    \end{equation}
\end{itemize}

Adicionalmente, se evaluó la eficiencia computacional de cada detector mediante 
las estadísticas generadas por el gestor de recursos Slurm, incluyendo: 

\begin{itemize}
    \item \textbf{Utilización de CPU/GPU} (número de núcleos)
    
    \item \textbf{Consumo de memoria RAM} (GB)
    
    \item \textbf{Tiempo de ejecución} (minutos)
\end{itemize}

\subsubsection{Herramientas de validación y análisis}

La visualización de las lecturas alineadas se realizó utilizando Genome-Wide (GW), 
un navegador genómico avanzado que permite navegar de forma instantánea y fluida 
a pesar del elevado volumen de los ficheros BAM (más de 100 GB cada uno). Este navegador 
facilita la curación manual mediante navegación ágil entre variantes anotadas en 
los VCF generados por los detectores de SVs, permitiendo validar los hallazgos 
(\textbf{Figura~\ref{fig:GW_UI}}).

Dado que el clúster carece de soporte para la ejecución de herramientas con 
interfaz gráfica, el directorio que contenía los ficheros BAM generados por las 
simulaciones se montó en un equipo local utilizando el protocolo SSHFS 
(\textit{Secure SHell FileSystem}).

Para el cálculo automatizado de las métricas descritas, se revisaron manualmente
todas las lecturas, considerando únicamente las SVs que pudieron ser reconstruidas exitosamente 
tras el alineamiento de las lecturas al genoma de referencia.    

\subsubsection{Procesamiento y visualización de resultados}

El procesamiento de los resultados, el cálculo comparativo de métricas y la 
generación de representaciones gráficas se llevaron a cabo mediante scripts 
desarrollados en R. El código completo está disponible en el repositorio 
asociado a este proyecto: 
\url{https://github.com/villena-francis/tfe_unir/tree/main/data/cluster_bmk/stats_plots.R}.